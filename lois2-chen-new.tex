%%「論文」,「レター」,「レター(C分冊)」,「技術研究報告」などのテンプレート
%% v3.3 [2020/06/02]

\documentclass[technicalreport]{ieicej}
%\usepackage[dvips]{graphicx}
\usepackage[dvipdfmx]{graphicx,xcolor}
\usepackage{float}
\usepackage[fleqn]{amsmath}
\usepackage{newtxtext}% 英数字フォントの設定を変更しないでください
\usepackage[varg]{newtxmath}% % 英数字フォントの設定を変更しないでください
\usepackage{latexsym}
\usepackage{listings}
\usepackage{xurl}
\usepackage{fancyvrb}
\usepackage{multirow}
\usepackage{array}
\usepackage[section]{placeins} % make sure all fig and table do not go to other sections
\usepackage{stfloats} % somehow avoid the possible blank page by fig or table
% \usepackage[normalem]{ulem}
% \useunder{\uline}{\ul}{}
%\usepackage{amssymb}


\renewcommand{\refname}{References} % To get english reference section heading
\renewcommand{\figurename}{Fig.} % To get english figure heading
\renewcommand{\tablename}{Table} % To get english table heading
\renewcommand\UrlFont{\rmfamily} % Fix url font

\lstset{%
    language={Python},
    basicstyle={\small},%
    identifierstyle={\small},%
    commentstyle={\small\itshape},%
    keywordstyle={\small\bfseries},%
    ndkeywordstyle={\small},%
    stringstyle={\small\ttfamily},
    frame={tb},
    breaklines=true,
    columns=[l]{fullflexible},%
    numbers=left,%
    xrightmargin=0zw,%
    xleftmargin=3zw,%
    numberstyle={\scriptsize},%
    stepnumber=1,
    showstringspaces=false,%
    numbersep=1zw,%
    lineskip=-0.5ex,%
    %moredelim=[is][\underbar]{_}{_},%
    keepspaces=true,%
    %escapechar=\@
}

% \jtitle{タイトル}
% \jsubtitle{}
\etitle{A Proposal of Speed-up Method for Fuzzy Search Process in Product Label Recognition System}
% \esubtitle{}
\authorlist{%
 \authorentry[chin-shiji@s.okayama-u.ac.jp]{陳 仕璽}{Shixi Chen}{okayama}
 \authorentry[funabiki@okayama-u.ac.jp]{舩曵 信生}{Nobuo Funabiki}{okayama}
 \authorentry[phvf8tn3@s.okayama-u.ac.jp]{坂上 正規}{Masaki Sakagami}{okayama}
 \authorentry[toshida.takashi@astrolab.co.jp]{土信田 高}{Takashi Toshida}{astrolab}
 \authorentry[suga@astrolab.co.jp]{菅 恒平}{Kohei Suga}{astrolab}
% \authorentry[メールアドレス]{和文著者名}{英文著者名}{所属ラベル}
}
\affiliate[okayama]{}
{Okayama University\hskip1em
Tsushimanaka 3-1-1, Okayama, 700--8530, Japan}
\affiliate[astrolab]{}
{Astrolab \hskip1em
Otemachi 2-6-2, Chiyoda, Tokyo, 100-0004, Japan}
%\affiliate[所属ラベル]{和文勤務先\\ 連絡先住所}{英文勤務先\\ 英文連絡先住所}
\jalcdoi{???????????}% ← このままにしておいてください

\begin{document}
% \begin{jabstract}
% %和文あらまし
% \end{jabstract}
% \begin{jkeyword}
% %和文キーワード
% \end{jkeyword}

\begin{eabstract}
    %background
    Recently, the {\em optical character recognition (OCR)} technology has been remarkably progressed to result in the high recognition rate due to the advancements of {\em deep learning techniques}. Besides, smartphones equipped with cameras have broadly spread among people around the world. As a result, the information acquisition from an object ravel using a smartphone camera can be a good choice as the easy and quick way.
    %problem to be solved
    However, the character recognition rate is still not $100\%$. Some characters are incorrectly recognized or missing in the recognition result. Then, if the correct characters to them have been stored as one record in a database, the {\em fuzzy search} can be applied to find the best matching record with the recognition result. Unfortunately, it will take unacceptably long time if the database has a lot of records, because it spends much longer time to process one record. 
    %contribution
    In this study, we propose a speed-up method for the fuzzy search by limiting the processing records into possible ones where the part of the characters in the record is matching. 
    %evaluation
    For evaluations, we collect 300 photos of object ravels and their character recognition results, and apply our proposal to them. The results show that the proposal reduces the CPU time from $???sec$ to $???sec$ by limiting the number of search records into 0.3\% of all the records. However, the record hit rate is also reduced from $???\%$ to $???\%$ whose improvement will be in future studies.
    
\end{eabstract}
\begin{ekeyword}
OCR, fuzzy search, regular expression, partial word matching
\end{ekeyword}
\maketitle

%+++++++++++++++++++++++++++++++++++++++++++++++
\section{Introduction}
    %background
    Recently, the {\em optical character recognition (OCR)} technology has been remarkably progressed to result in the high recognition rate due to the advancements of {\em deep learning techniques}. Besides, smartphones equipped with cameras have broadly spread among people around the world. As a result, the information acquisition from an object ravel using a smartphone camera can be a good choice as the easy and quick way.

    %problem to be solved
    However, the character recognition rate is still not $100\%$. Some characters are incorrectly recognized or missing in the recognition result. Then, if the correct characters to them have been stored as one record in a database, the {\em fuzzy search} can be applied to find the best matching record with the recognition result. Unfortunately, it will take unacceptably long time if the database has a lot of records, because it spends much longer time to process one record. 

    %contribution
    In this study, we propose a speed-up method for the fuzzy search by limiting the processing records into possible ones where the part of the characters in the record is matching. 
    もっと詳しく書いてください.

    %evaluation
    For evaluations, we collect 300 photos of object ravels and their character recognition results, and apply our proposal to them. The results show that the proposal reduces the CPU time from $???sec$ to $???sec$ by limiting the number of search records into 0.3\% of all the records. However, the record hit rate is also reduced from $???\%$ to $???\%$ whose improvement will be in future studies.
    もっと詳しく書いてください.

    %contents
    The rest of this paper is organized as follows:
    Section \ref{sec:preliminary} shows preliminary technologies to this study.
    Section \ref{sec:proposal} presents the speed-up method.
    Section \ref{sec:evaluation} shows the evaluation results.
    Finally, Section \ref{sec:conclusion} concludes this paper with future works.


%+++++++++++++++++++++++++++++++++++++++++++++++
\section{Preliminary}
\label{sec:efp}
    In this section, we introduce preliminary technologies of our study in this paper.

    \subsection{Levenshtein Distance}
        The Levenshtein distance is a string metric for measuring the difference between two sequences. Informally, the Levenshtein distance between two words is the minimum number of single-character edits (insertions, deletions or substitutions) required to change one word into the other.

        【Replace this part with something from another paper to increase cites】

        (\url{https://en.wikipedia.org/wiki/Levenshtein_distance})

    \subsection{FuzzyWuzzy}
        FuzzyWuzzy is a Python library useing Levenshtein Distance to calculate the differences between sequences. It relies on the Difflib module.

        【Replace this part with something from another paper to increase cites】

        (\url{https://chairnerd.seatgeek.com/fuzzywuzzy-fuzzy-string-matching-in-python/})
        
        (https://github.com/seatgeek/fuzzywuzzy)

%+++++++++++++++++++++++++++++++++++++++++++++++
\section{Proposal of Speed-up Method}
\label{sec:algorithm}
    In this section, we propose the speed-up method for the Fuzzy search process in the product label recognition system.

    \subsection{System Overview}
        Our system is that user take a picture and give it to the system, the system will give the correct model of the product with the help of a prepared master DB.

        In the proposal, first, optical character recognition (OCR) is used to convert the picture into text. Second, we use a regular expression to filter and extract all alphanumeric strings which have possibility to be model text from the OCR recognition result. Then, we split the alphanumeric strings into two halves, and use partial word matching to pick model texts containing either half from the master database, which is pre-prepared and contains a large number of product models. Finally, we do fuzzy search in all the models picked, to find model texts most similar to any of the alphanumeric strings, which are the most possible real model. 
        
    \subsection{System Input}
    \label{sec:algorithm.ocrregex}
        Our input are picture of product labels and a prepared master database.

        \subsubsection{Product Labels}
        Our picture should be taken from front of a label, and have a good contrast, etc\dots

        \subsubsection{Master Database}
        We've got a Master DB containing a lot of prepared model data.

        [An example of DB data]
    
    \subsection{System Output}
        Our purpose is to output the model of the product in the picture. The output is a list sorted by matching rate from high to low. The most possible model is the first one.

        [An example of result list]

    \subsection{Conventional Method}
        In the conventional method, we use OCR, regular expression matching, and fuzzy search to get the result list.

        \subsubsection{OCR}
        First, OCR is used to convert the picture into text.
        
        \subsubsection{Alphanumeric Strings Matching}
        Second, we use a regular expression to filter and extract all alphanumeric strings which have possibility to be model text from the OCR recognition result.

        \subsubsection{Fuzzy Search in Master DB}
        Then, we do fuzzy search in all theMaster DB, to find model texts most similar to any of the alphanumeric strings, which are the most possible real model. 

    \subsection{Proposal of Speed-up Method}
        Method above is slow, we have made it faster with this method.

        \subsubsection{Half Divided String}
            Before doing fuzzy search, we devide the matched Alphanumeric Strings into two halves.

        \subsubsection{Partial Word Matching}
            We use the divided string to mathc in master DB. Like in table \ref{table:half_matching}. We use partial matching rather than forward or backward matching so that we can avoid OCR text missing.

            \begin{table}[tb]
                \caption{Half divided string and its matching results}
                \label{table:half_matching}
                \begin{center}
                    \begin{tabular}{c|c|c}
                    \Hline
                    origional label text & \multicolumn{2}{c}{MX1234567} \\ 
                    \hline
                    OCR result & \multicolumn{2}{c}{MX12345{\em b}7} \\ 
                    \hline
                    half divided strings & \_\_MX123\_\_ & \_\_45{\em b}7\_\_ \\
                    \hline
                    matching results & \begin{tabular}{c}MX123\underline{4567}\\MX123\underline{5678}\\\underline{P}MX123\underline{5678}\\...\end{tabular} & (no result)) \\
                    \Hline
                    \end{tabular}
                \end{center}
            \end{table}
        
%+++++++++++++++++++++++++++++++++++++++++++++++
\section{Evaluation}
\label{sec:evaluation}
    In this section, we evaluate the proposal. 

    \subsection{Photo Source}
        We took more than 300 photos to perform the test, all of them match the request.

    \subsection{Regular Expression being Used}
        The regular expression we use:

        \begin{center}
        \begin{BVerbatim}
[ :]*(([a-zA-Z0-9\-\/\(\)]{4,} (?=[ a-zA-Z
0-9\-\/\(\)]*[0-9])[ a-zA-Z0-9\-\/\(\)]+)|
((?=[a-zA-Z0-9\-\/\(\)]*[0-9])[a-zA-Z0-9\-
\/\(\)]{4,}))[ ,.]*
        \end{BVerbatim}
        \end{center}
    
    \subsection{Result}
        Before is Non-Limit Fuzzy Search. After is Fuzzy Search Limited by Partial Word Matching.
    
        We conducted 300 searches in a master database with 550,000 records. The time consumed is shown in the table \ref{table:methods_compare}.
    
    \begin{table*}[t]
        \caption{Performance result using different searching methods [DATA NEEDS CORRECTION]}
        \label{table:methods_compare}
        \begin{center}
            \begin{tabular}{c|cc}
            \Hline
            search method &
                \begin{tabular}{c}Non-Limit\\Fuzzy Search\end{tabular} &
                \begin{tabular}{c}Fuzzy Search\\Limited by Partial\\Word Matching\end{tabular} \\ 
            \hline
            1st candidate in matching list was correct &
                 290 & \\ 
            \hline
            2nd candidate in matching list was correct &
                 9 & \\ 
            \hline
            Recognition Rate &
                 99.67\% & \\ 
            \hline
            Average Search Time &
                 120s & 0.13s \\ 
            \hline
            Maximum Search Time &
                 180s & 13s \\ 
            \hline
            Average Search Range (DB Record Amount) &
                 550,000 & 3,000 \\
            \Hline
            \end{tabular}
        \end{center}
    \end{table*}
    
    Among the Fuzzy Search limited by partial word matching results, the several experiments in which the time-consuming is significantly higher than the average time-consuming are as in the table \ref{table:slowest_rec}.

    \begin{table*}[t]
        \caption{Performance result using different searching methods [DATA NEEDS CORRECTION]}
        \label{table:slowest_rec}
        \begin{center}
            \begin{tabular}{c|c|c|c|c|c|c}
            \Hline
            No. &
                Real Product Model &
                \begin{tabular}{c}Query Alphanumeric\\String (OCR result)\end{tabular} &
                \begin{tabular}{c}Limiting Time\\(Partial Word Matching\\Costed Time)\end{tabular} &
                \begin{tabular}{c}Fuzzy Search\\Costed Time\end{tabular} &
                \begin{tabular}{c}Fuzzy Search Range\\(DB Record Amount)\end{tabular} &
                \begin{tabular}{c}Total Costed Time\end{tabular} \\ 
            \Hline
            1 & & & & & & \\ 
            \hline
            2 & & & & & & \\ 
            \Hline
            \end{tabular}
        \end{center}
    \end{table*}
    
            
%+++++++++++++++++++++++++++++++++++++++++++++++
\section{Conclusion}
\label{sec:conclusion}
    It can be seen that using Fuzzy Search Limited by Partial Word Matching can effectively reduce the time into the acceptable range while increasing the accuracy rate to be close to that of Non-Limit Fuzzy Search. With this method, we can take a balance between searching speed and mistake prevention for OCR text, and successfully limit the search range into several thousand (about 0.3\% of all data amount). In most cases the search result would come out in only less than 1 second which is acceptable for users to wait. This method turns out to be possible to put into practice when deploying into web services.
    
    It can be seen from the experimental data that for rare cases, after the limitation of character recognition, the search range is still large.
    
    We consider that, for each record in the master database, it not only contains the model text information of the product, but also its brand or manufacturer information. Before doing Fuzzy Search, we can also use other recognition methods to match and analyze the text contained in the OCR results to determine the brand of the product, and use brand information matching to further limit the search range of Fuzzy Search for model texts, which may further improve searching performance.
    

%+++++++++++++++++++++++++++++++++++++++++++++++
%\bibliographystyle{sieicej}
%\bibliography{myrefs}
\begin{thebibliography}{99}% 文献数が10未満の時 {9}
    \bibitem{popularity}
    \url{https://www.spectrum.ieee.org/at-work/tech-careers/top-programming-language-2020}

    \bibitem{Funabiki13} 
    N. Funabiki, Y. Matsushima, T. Nakanishi, and N. Amano, "A Java programming learning assistant system using test-driven development method," \emph{IAENG Int. J. Comput. Sci.}, vol. 40, no.1, pp. 38-46, Feb. 2013.
    
    \bibitem{Zaw15}
    K. K. Zaw, N. Funabiki, and W.-C. Kao, "A proposal of value trace problem for algorithm code reading in Java programming learning assistant system," \emph{Inf. Eng. Express}, vol. 1, no. 3, pp. 9-18, Sep. 2015.
    
    \bibitem{Funabiki17} 
    N. Funabiki, Tana, K. K. Zaw, N. Ishihara, and W.-C. Kao, "A graph-based blank element selection algorithm for fill-in-blank problems in Java programming learning assistant system, \emph{IAENG Int. J. Comput. Sci.}, vol. 44, no. 2, pp. 247-260, May 2017.
    
    \bibitem{cup}
    CUP, \url{http://czt.sourceforge.net/dev/java-cup/manual.html}
    
    \bibitem{scope}
    Scope, https://www.codesdope.com/blog/article/scope-of-variables-in-c/
    
    \bibitem{example}
    \url{http://www.codebind.com/c-examples/}
    
    \bibitem{triangle}
    \url{https://brilliant.org/wiki/pascals-triangle/}
    
    \bibitem{related1}
    A. Kashihara, A. Terai, and J. Toyota, "Making fill-in-blank program problems for learning algorithm, " \emph{in Proc. Int. Conf. Comput. Edu.}, pp. 776-783, 1999.
    
    \bibitem{related2}
    J. Shinkai, Y. Hayase, and I. Miyaji, "A study of generation and utilization of fill-in-the-blank questions for programming education on Moodle, " \emph{IEICE Technical Report}, vol. 110, no. 263, pp. 7-10, 2010.
    
    \bibitem{related3}
    K. Terada, Y. Watanobe, "Automatic generation of fill-in-the-blank programming problems, " \emph{in Proc. IEEE Int. Symp. Embed. Mult./Many-core Sys.-on-Chip (MCSoC)}, pp. 187-193, 2019.
\end{thebibliography}

\end{document}

